\chapter{Conclusions}
\label{chap:conclusion}

After a brief introduction, we discussed all the necessary knowledge needed
to understand this thesis in \cref{chap:background}, and finally we presented
how \f{} works, by thoroughly examining its inner components in \cref{chap:implementation}.

The most important concept to recall about this framework is its goal, and how it tries to achieve it.
\f{} wants to enhance the serverless programming model through its meta-programming
features. In fact, its core annotations operate very differently from each other,
as they are related to distinct domains, but they still can be combined
to enrich the development experience.
Thus, \f{}'s users are not only inheriting all the
benefits offered by serverless architectures,
but they are also accessing all its powerful code
transformations that can save precious development time.

\f{} was also designed to help migrate existing
monolithic codebases into serverless ones through an incremental approach,
meaning that users do not need to convert their entire project, but they
can partially adopt this new computing paradigm by iteratively annotating which
components should be deployed as independent functions.
Additionally, it is crucial to understand that the code
emitted by \f{} is not set in stone. Instead, it can be actively modified
and even used as a starting point of a serverless codebase.
\f{} should be viewed as a coding assistant rather than a fully comprehensive tool to write serverless functions.

\f{}'s most notable limitation is the lack of more core annotations,
which stems from the substantial time investment required
to establish the framework's foundations, reflect on valuable transformers and write their implementation.
Still, users are encouraged to create their custom annotations
tailored to address their specific use cases, which can include scenarios
that cannot be foreseen by \f{} due to their unique nature.

Another important decision to discuss is our rationale behind selecting \textit{TypeScript}.
Many other programming languages offer meta-programming compile-time capabilities,
like \textit{Rust} macros, \textit{C++} templates or \textit{Zig} comptime.
Such systems are far superior to our annotations, yet,
these low-level languages are not suitable for the high-level programming
that the serverless functions are supposed to entail, in fact most of these languages
do not have the runtime support from any major platform provider.
Moreover, even high-level languages with similar compile-time features, like \textit{Scala},
demand to be monitored with caution, because of their heavy
runtime (i.e., \textit{JVM} for \textit{Scala}) initialization spin-ups
that may cause expensive cold starts.

\textit{TypeScript}, enriched with our annotations, perfectly fits
the serverless model as it runs, after its strict static type checks,
on \textit{Node.js}, which is a well-suited lightweight environment
designed to handle distributed services, as discussed in \cref{sec:node}.

\section{Comparison with previous studies}

One of the features offered by \f{} is the fact that it can port
an existing monolith to a serverless platform as shown in \cref{sec:example}.
This process, named ``FaaSification'', has captured the attention of researchers
who have explored and presented various solutions to this challenge.

The first attempts were developed for \textit{Python} and \textit{Java} monoliths,
with the tools being named respectively \textit{Lambada} \cite{lambada} and \textit{Termite} \cite{termite}.
They are fairly limited, as their emitted code cannot correctly process inputs,
and they do not support the use of global variables within the serverless functions.

Instead, let us focus on the technologies built around \textit{JavaScript}:

\paragraph{\textbf{Node2Faas} \cite{node2faas}}
It was the first \textit{JavaScript} ``FaaSifier''.
It converts methods of the \textit{Node.js} monolith into serverless
functions and replaces their bodies with an API call to the target FaaS system.
This tool has many weaknesses: most notably, it cannot resolve either code
or package dependencies, which is a considerable constraint since many
cloud functions work by using third-party services.

\paragraph{\textbf{DAF} and \textbf{M2FaaS} \cite{daf, m2faas}}
They were built by the same team of researchers, the latter, \textit{M2FaaS},
represents an improved iteration of the former.
They introduce a concept similar to our annotations, through which users
can mark arbitrary parts of their code with the dependencies needed to be deployed.
The outcome is a serverless monolithic application hybrid. 
However, due to technical limitations, the development effort to write their marking constructs
is remarkable, as users are compelled to write configuration details inside their business logic.

\paragraph{\textbf{FaaSFusion} \cite{faasfusion}}
This is the closest work to \f{}.
\textit{FaaSFusion}'s users also mark their functions with annotations
and some associated code transformations occur.
This framework brings infrastructure-as-code concepts into the function source,
but this approach has evident shortcomings since it heavily relies on heuristics for its foundations.
For example, it offers an annotation named \annotation{@Warmup} which
injects an algorithm to avoid cold starts by periodically pinging the function
through a \textit{CloudWatch} event:
however, platform providers have suggested avoiding these so-called ``warmers'',
because they are mostly ineffective, break serverless principles, and can increase costs very quickly.

The main obstacle that these tools do not overcome is that they are
not designed to handle codebases beyond the trivial examples
which they present to support their cases.
In comparison, \f{} is very powerful, and its scope is broader.

\section{Future directions}

We believe \f{} could be improved further by expanding its features
or by building tools built around it.
We close this thesis by listing a few of the enhancements \f{} could receive.

\subsection{More annotations}

\f{} should have more core annotations.
Also, while the existing annotations currently do not exhibit conflicts with each other,
it is prudent to anticipate potential conflicts that could arise when new annotations are introduced.
Consequently, incorporating new constraints becomes essential to prevent such
conflicts and ensure the smooth coexistence of annotations within the framework.

\f{} should add support for marking not just top-level functions, but also other AST nodes.
This feature would enrich the framework further, as users would have more freedom
to change the emitted code. These new annotations may be inside the function itself,
marking nodes inline, or even at the source level. Additionally, they may be related to a particular annotation,
thus, creating sets of domain-specific annotations.
For example, \annotation{\$Fixed}, whose scope is related to ``FaaSification'',
could have \textit{children} annotations that would issue a warning if used under other top-level annotations.
\begin{lstlisting}[language=javascript]
/** $\dollar$Fixed */
export async function foo() {
  const data = await query()
  // new inner annotation, which changes parts of the emitted code
  /** $\dollar$StatusCode(code: 201) */
  return data
}
\end{lstlisting}

\subsection{Formalization}

Many parts of this framework should be formalized, similarly to the work conducted by Kallas et al. \cite{formalization},
as this would give solid foundations to \f{}:

\begin{itemize}
  \item Annotation syntax and semantics, including defining
    the structure of annotations, allowed parameters, and their meanings,
    because we only presented a preliminary insight in \cref{sec:annotations}.

  \item Composition semantics to spot potential conflicts while using multiple annotations.

  \item Transformations algorithms to ensure correctness, predictability,
    and consistency in the compilation process, given some notion of behavioral correspondence.
\end{itemize}

\subsection{Linter}

Currently, there is no standalone linter or set of rules for existing ones
that captures serverless principles and enforces them on projects.
A new linter may analyze the code, flag architectural smells,
and warn users whenever they are breaking best practices.
For example, the linter may issue an error if there is a serverless function invoking another one,
as this could lead to increased costs, more debugging complexity, and a breach of the isolation principle.

Furthermore, a new linter could be built specifically for \f{},
flagging errors and warnings when a developer misuses annotations.
Linting may take place either before the transpilation process or afterward, targeting the emitted directory.
