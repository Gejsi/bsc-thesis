\chapter{Introduction}
\label{chap:introduction}
% Briefly introduce the topic of serverless computing and its growing significance in modern software development.
% Mention the challenges and complexities developers face when working with serverless architectures.
Modern software development requires complex tooling to handle
all the concepts which surround the actual code and business logic.
Besides the actual program implementation, there are many components to consider
when thinking about all the moving parts needed to make software work
in today's dynamic environments, which are supposed to be scalable by design.

Traditionally, software was written in monolithic architectures where
the program's components are tightly coupled and tend to create
large codebases, as they provide heterogeneous functionalities under the same interface.
This paradigm, despite being the oldest, still has some benefits that
draw developers' attention especially in the early stages of the development lifecycle:
indeed, these architectures furnish the maximum amount of control over the infrastructure,
are easier to write because they do not need advanced orchestration mechanisms,
therefore simplifying testing and debugging.
Yet, the drawbacks of this paradigm were understood even in the past:
a case in point is the \textit{Unix} community, which moved away from it,
preferring a modular approach instead. Monoliths are hard to maintain, deploy,
adapt to changing requirements and technologies, and
become rigid quickly as reshaping its components may involuntarily cause a chain of problems.

On the other hand, a modular approach, where features are developed in
independent software units, gained popularity in the cloud services realm.
These so-called microservices architectures arrange
programs into collections of orchestrated, loosely coupled, fine-grained services
communicating through networking protocols such as HTTP.
A microservice is a self-contained service, organized around a specific set
of needed capabilities, that lets developers easily implement a layered architecture
using different technologies (e.g., programming languages, databases, etc.).
This distributed architectural pattern overcomes the shortcomings of monoliths,
as its modular nature makes it scalable, easily deployable, and allows different teams
to work together without the risk to take down the entire structure when a minor issue comes up.

Still, microservices are not the most efficient way of handling cloud computing,
since they are long-running processes that need to be carefully planned and managed
by configuring the underlying infrastructure. To resolve this operational costs,
serverless architectures were recently created.

The serverless pattern promises to let its users focus simply on the business logic,
by shifting away to the cloud platform provider the infrastructure management,
resources allocation and provisioning, thus enabling computing units to autoscale on a pay-as-you go manner.
In this model, users write the code inside single-purpose units (i.e., programming languages' functions),
so they are even smaller than microservices in responsibility,
and they set which event will trigger its invocation.
This event-driven model of dispatching actions is very efficient, in fact,
once a function is triggered, the cloud vendor executes it in an isolated,
secure and short-lived environment.
Consequently, users are only billed for the precise resources consumed during this execution period.


\section{Objectives} % *incremental adoption*

Lorem ipsum dolor sit amet, qui minim labore adipisicing minim sint cillum sint consectetur cupidatat.
