\begin{center}
  \textsc{Sommario}
\end{center}
%
\noindent
%
\begin{otherlanguage}{italian}
Con il termine \textit{serverless} si indica un nuovo modello architetturale nel campo del cloud computing,
caratterizzato dall'esecuzione distribuita, scalabile e basata sugli eventi dei programmi,
con la peculiarità che i costi sono proporzionalmente correlati al consumo effettivo delle risorse.
Gli sviluppatori scrivono il codice in unità software indipendenti, chiamate \textit{funzioni serverless},
e affidano ai fornitori delle piattaforme la complessa gestione dell'infrastruttura sottostante.
Dopo aver esplorato questo paradigma, introduciamo \f{},
un framework che arricchisce il ciclo di sviluppo delle architetture serverless fornendo agli sviluppatori
nuove costrutti di meta-programmazione, dette \textit{annotazioni},
che dotano le funzioni serverless di attributi distinti,
e consentono di modellarne il comportamento e le caratteristiche per adattarle alle specifiche esigenze dell'applicazione.
\f{} permette di sfruttare tutti i vantaggi del serverless
senza sacrificare la compatibilità con progetti già consolidati,
poiché è in grado di convertire monoliti esistenti, scritti in \textit{TypeScript}, in architetture serverless.
\end{otherlanguage}

\newpage


\begin{center}
  \textsc{Abstract}
\end{center}
%
\noindent
%
Serverless computing is a new cloud architectural model that promises
to execute programs in a distributed, autoscaling, event-driven, and pay-as-you go manner.
Developers write the code in self-contained software units, called ``serverless functions'',
and they let the cloud vendors manage the complex infrastructure underneath.
After exploring this architectural model, we introduce \f{}, which is a framework that enriches the development
lifecycle of this model by providing the developers with new
meta-programming constructs, named \textit{annotations},
that imbue serverless functions with distinct attributes,
augmenting their behavior and characteristics to align with specific application requirements.
\f{} offers developers an avenue to harness the merits of serverless
without sacrificing compatibility with established codebases, as \f{} can convert
existing \textit{TypeScript} monoliths into serverless architectures.

\newpage
\
\thispagestyle{empty}
\cleardoublepage
