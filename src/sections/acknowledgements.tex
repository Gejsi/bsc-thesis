\cleardoublepage % start the page at right
\phantomsection
\addcontentsline{toc}{chapter}{Ringraziamenti}
\chapter*{Ringraziamenti}
\pagestyle{empty}
\thispagestyle{empty}
\markboth{Ringraziamenti}{Ringraziamenti}

Questa sezione è necessariamente scritta in italiano,
perché non bisogna scordarsi che anche dietro ad un documento come questo
che può sembrare distaccato, c'è dietro una persona che ha lavorato duramente per tre anni.

Voglio ringraziare prima di tutto i miei due relatori che mi hanno
permesso di presentare questo mio umile progetto e mi hanno lasciato piena
libertà decisionale su ogni aspetto. Le loro indicazioni sono state fondamentali:
ringrazio il prof. Davide Sangiorgi che con i suoi consigli
mi ha fatto da mentore in questi anni, e ringrazio il prof. Saverio Giallorenzo
che, da grande stachanovista quale è, mi ha seguito e guidato attivamente durante
tutto lo sviluppo. Li ringrazio soprattutto perché sono tra le persone più competenti
che io abbia mai incontrato e perché mi hanno introdotto a quei campi di questa scienza
che mai avrei pensato potessero suscitare tanto interesse in me.
Inoltre, ringrazio anche il Dr. Matteo Trentin ed il Dr. Giuseppe De Palma,
che oltre ad essere due dottorandi brillanti e pieni di idee nuove, mi hanno sempre dato una mano.

Ringrazio la mia famiglia per non avermi messo nessun tipo di pressione,
e per avermi accolto sempre a casa con gioia. Mi hanno reso un uomo maturo ed indipendente.

Ciò che mi rimarrà di questi tre anni a Bologna, più che le nozioni,
saranno le persone e la valanga di ricordi che ho creato con i miei amici.
Ho conosciuto persone incredibili, nel bene o nel male, che solo Bologna poteva offrire.
Il gruppo che si è venuto a formare era pieno di personaggi assurdi a cui auguro il meglio.
Ringrazio Tommaso, Gabriele, Simone, Ilaria, Francesco, Greta, Arianna e tutti gli altri
che non riesco a citare perché sono troppi.

Infine, ringrazio i miei migliori amici Luca, Giovanni e Giulio,
con cui ho avuto la fortuna di condividere la casa per tre anni.
Sarà impossibile trovare dei coinquilini migliori, e sarà impossibile
scordarsi di tutte le risate che ci siamo fatti.

\clearpage
